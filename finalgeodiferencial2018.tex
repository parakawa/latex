\documentclass{article}
  \usepackage{amsfonts}
  \usepackage[utf8]{inputenc}
  \begin{document}
  
  \begin{center}
    \bfseries\large
    UNIVERSIDAD NACIONAL MAYOR DE SAN MARCOS
    
    FACULTAD DE CIENCIAS MATEMÁTICAS
    
    Escuela Profesional de Matemática
    
    Introducción a la Geometría Diferencial 
    
    Examen Final \qquad \  28 Jun 2018
  
    \bigskip
  
  \end{center}
  
  \noindent
  \begin{enumerate}
      \item 
      \begin{enumerate}
          \item
          Hallar la ecuación de la evolvente de la curva descrita por \break $\overline{R}(\lambda)=(acos\lambda,asen\lambda, b\lambda)$, a y b constantes. \quad (3 ptos)
          \item
          Demostrar la siguiente propiedad de las curvas de Bertrand: El producto de las torsiones de las curvas $\Gamma \ y \ \Gamma ^{*}$ en puntos correspondientes es constante. \quad (3 ptos)
          \item
          Demostrar que el plano osculador tiene una curva, en P, un contacto de tercer orden, por lo menos si y solamente si la curvatura o la torsión se anulan en P. \quad (3 ptos)
      \end{enumerate}
      \item
      Se considera una circunferencia de radio 2a y centro en el punto F1 y el punto F2 en el interior de esta circunferencia, que se encuentra a la distancia 2c respecto al centro. El punto arbitrario A de la circunferencia se une con el punto F2 mediante un segmento, y a través de la mitad de éste se traza una perpendicular L. Hallar la envolvente de la familia de semejantes perpendiculares. \quad (5 ptos)
      \item
      Demostrar: En el espacio $\mathbb{R}^3$, la gráfica de una ecuación en dos de las tres variables $x, y \y z $ es un cilindro cuyas generatrices son paralelas al eje asociado con la variable faltante, y cuya directriz es una curva en el plano asociado con las dos variables que aparecen en la ecuación. \quad (4 ptos)
      \item
      Se considera la superficie de ecuaciones paramétricas $x=u+v, y=u-v, z=4uv$, para $-\infty<u<\infty$, $-\infty<v<\infty$.
      \begin{enumerate}
          \item Efectuar el cambio de parámetros dado por $u=\frac{\alpha(cosh\beta+senh\beta)}{2}$, \break $v=\frac{\alpha(cosh\beta-senh\beta)}{2}$
          \item ¿Conserva el cambio de parámetros el carácter de los puntos de la superficie? \quad (2 ptos)
      \end{enumerate}
  \end{enumerate}
  \end{document}