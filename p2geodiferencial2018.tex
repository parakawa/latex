\documentclass{article}
\usepackage{amsfonts}
\usepackage[utf8]{inputenc}
\begin{document}

\begin{center}
  \bfseries\large
  UNIVERSIDAD NACIONAL MAYOR DE SAN MARCOS
  
  FACULTAD DE CIENCIAS MATEMÁTICAS
  
  Escuela Profesional de Matemática
  
  Introducción a la Geometría Diferencial 
  
  Práctica Calificada Nª2 \qquad \  21 Jun 2018

  \bigskip

\end{center}

\noindent

\begin{enumerate}
  \item 
  Hallar las evolutas de la curva $x=sen\lambda,  y=\lambda+1,  z=cos\lambda \qquad$  (4 ptos)
  \item 
  \begin{enumerate}
    \item
    Hallar el orden de contacto de la curva $x=e^{\lambda} , y=\lambda^3, z=e^{-\lambda}$ con la curva de ecuaciones $y^3+xz=1, y+2x^2z^2=2$ \qquad (2 ptos)
    \item
    Hallar la ecuación de la esfera osculatriz de la curva $x=acos\lambda, y=asen\lambda, z=b\lambda$ con a y b constantes, $\lambda \in (-\infty,\infty)$, en el punto $(a,0,0)$ \qquad (4 ptos)
  \end{enumerate}
  \item 
  En el plano se da un ángulo recto. Se trazan todas las rectas posibles que cortan en este ángulo triángulos cuya área es 2. Hallar la envolvente de todas estas rectas. \qquad (4 ptos)
  \item
  \begin{enumerate}
      \item 
      Hallar la superficie de revolución que se obtiene al girar alrededor de la recta $x=y=z$ la curva de ecuaciones $y=x^2, y+z=0$ \quad (4 ptos)
      \item
      Sea $f:D\subset \mathbb{R}^2\rightarrow\mathbb{R}^3$ una función diferenciable. Estudiar si la superficie $z=f(x,y)$, para $(x,y)\inD$ tiene puntos singulares. \quad (2 ptos)
      
  \end{enumerate}
\end{enumerate}

\end{document}