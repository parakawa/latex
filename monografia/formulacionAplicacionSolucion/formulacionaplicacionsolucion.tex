%%%%%%%%%%%%%%%%%%%%%%%%%%%%%%%%%%%%%%%%%%%%%%%%%%%%%%%%%%%%%%%%%%%%%%%%%%%%%%%%%%%%
\section{Ecuación de Fisher-Kolmogorov (FK) en oncología.}

Tal como se mencionó en la \textit{Introducción}, existe una multitud de estudios y modelos matemáticos relacionados al cáncer. El objetivo ahora, es establecer las condiciones hipotéticas para el modelo del crecimiento de un tumor cerebral y ver cómo se acondiciona a la ecuación de Fisher-Kolmogorov (\ref{deduccion4}) , estableciendo así un modelo matemático la cual estudiaremos empleando métodos numéricos\\

Trabajaremos con condiciones de frontera \textit{(Neumann)}, puesto que se asume que el \textit{área es cerrada}, esto para asegurar que no existirá migración de células a través del dominio establecido. Entonces:

\begin{equation}
	\label{oncologia1}
	\dfrac{\partial u(0,t)}{\partial x}=\dfrac{\partial u(L,t)}{\partial x}=0
\end{equation}

\vspace{0.5cm}
donde $L:$ Longitud del dominio. Nótese ademas que solo estamos trabajando en $ \Omega \subset \mathbb{R}$\\

Para esclarecer la frase \textit{área cerrada} podemos mencionar que la principal razón de porqué elegimos este modelo y condiciones de frontera para el estudio de un tumor cerebral(glioma maligno) es que este fenómeno en particular casi siempre presenta ausencia de metástasis.

\begin{obs}
	\upshape{La metástasis ocurre cuando las células cancerosas se separan del tumor original(primario) y viajan a través de la sangre hacia otros órganos y/o tejidos, y forman nuevos tumores. La \textit{ausencia de Metástasis en el tumor cerebral} significa que si se tiene un tumor cerebral, este raramente se expandirá hacia otros órganos, podemos interpretar esto entonces como si tuviéramos una cierta cantidad de puntos en un dominio $ \Omega$ y este sufrirá escasa alteración, al menos por migración de células en todo el dominio.
	}
\end{obs}

También debemos notar que no se está tomando en cuenta la posición de las células en un lugar específico(en qué parte del cerebro), en la realidad esto suele ocasionar grandes variantes, pues nos daría información acerca de la capacidad de invasión y aparente borde del tumor.\\

No obstante, como en todos los tumores, los aspecto biológicos y clínicos de los gliomas, son complejos y los detalles de sus crecimiento espacio-temporal todavía no se entienden bien. Para construir estos modelos buscamos describir los aspectos más básicos  de los gliomas, haciendo uso de la ecuación de Fisher-Kolmogorov para describir la dinámica espacio-temporal de una densidad de células cancerígenas. En la ecuación \eqref{deduccion4} el factor $f(u)$ modela la proliferación, es decir, otorga información acerca del nacimiento y muerte de células cancerosas. Por otro lado, el factor $ d{ u }_{ xx }$ surge a partir de que las células crecen lo suficiente y luego migran, esto es, se difunden.\\

En conclusión, la \textbf{ecuación de Fisher-Kolmogorov (FK)} para el modelo matemático que estima el crecimiento de las células cancerígenas en un tumor cerebral(glioma maligno) es: 

\begin{equation}
	\label{eq: jisus}
	\displaystyle \begin{cases} { u }_{ t }=d{ u }_{ xx }+\rho u(A-u) \\ u(x,0)=g(x) \\ { u }_{ x }(0,t)={ u }_{ x }(L,t)=0 \end{cases}
\end{equation}
%%%%%%%%%%%%%%%%%%%%%%%%%%%%%%%%%%%%%%%%%%%%%%%%%%%%%%%%%%%%%%%%%%%%%%%%%%%%%%%%%%%%%%%%%%%%%%
\section{Esquemas de diferencias finitas para el método de Fisher.}\label{diferenaciasfinitas}

En esta sección estudiaremos los esquemas en diferencias finitas para el sistema \eqref{eq: jisus}. Usaremos las fórmulas de aproximaxión que se pueder ver en el anexo. Partimos de la ecuación:

	$$u_{_{t}}=du_{_{xx}}+\rho(A-u)u$$

Usando diferencias centradas de segundo orden en la variable espacial para aproximar la derivada de segundo orden obtenemos: 

$$u_{_{xx}}(x_{_{i}},t_{_{j}})\simeq\frac{w_{_{i+1,j}}-2w_{_{i,j}}+w_{_{i-1,j}}}{h^{2}}$$

y usando ahora diferencias progresivas en el tiempo para aproximar la derivada de primer orden producimos la siguiente aproximación:

$$u_{_{t}}\simeq\frac{w_{_{i,j+1}}-w_{_{i,j}}}{k}$$

Sustituyendo esas aproximaciones en la ecuación obtenemos la ecuación aproximada:

$$\dfrac{w_{_{i,j+1}}-w_{_{i,j}}}{k}=d\left(\dfrac{w_{_{i-1,j}}-2w_{_{i,j}}+w_{_{i+1,j}}}{h^{2}}\right)+\rho(A-w_{_{i,j}})w_{_{i,j}}$$

Multiplicando ambos lados de la igualdad por $k$, se obtiene

$$w_{_{i,j+1}}-w_{_{i,j}}=\frac{dk}{h^{2}}(w_{_{i-1,j}}-2w_{_{i,j}}+w_{_{i+1,j}})+k\rho(A-w_{_{i,j}})w_{_{i,j}}$$

De donde se tiene:

\begin{equation}
	w_{_{i,j+1}}=w_{_{i,j}}+\sigma(w_{_{i-1,j}}-2w_{_{i,j}}+w_{_{i+1,j}})+k\rho(A-w_{_{i,j}})w_{_{i,j}}
	\label{iteracionsinneumann}
\end{equation}


donde $\sigma=\frac{dk}{h^{2}}$; $h=\frac{L}{m}$ y $k=\frac{T}{n}$ son los tamaños de paso siendo $L$ la longitud del intervalo espacial, $T$ el tiempo final, $m$ y $n$ la cantidad de intervalos en las que se divide el intervalo espacial y temporal respectivamente. Se está considerando $i=0,...,m$ y $j=0,...,n$ \\

Además de la condición de Neumann para $i=0$ aplicando una aproximación progresiva se tiene que:

\begin{equation}
	w_{_{0,j+1}}=w_{_{0,j}}+\frac{dk}{h^{2}}(2w_{_{1,j}}-2w_{_{0,j}})+k\rho(A-w_{_{0,j}})w_{_{0,j}}
	\label{neumannizquierdo}
\end{equation}

Por otro lado haciendo una aproximación centrada de la condición de  Neumann para $i=n$ se tiene:

\begin{equation}
	w_{_{m,j+1}}=w_{_{m,j}}+\frac{dk}{h^{2}}(2w_{_{m-1,j}}-2w_{_{m,j}})+k\rho(A-w_{_{m,j}})w_{_{m,j}}
	\label{neumannderecha}
\end{equation} 

De las ecuaciones \eqref{iteracionsinneumann}, \eqref{neumannizquierdo}, \eqref{neumannderecha} y de la condición inicial tenemos la fórmula de recursividad para aproximar o dar solución numérica al sistema \eqref{eq: jisus}

\subsection{Fórmula recursiva para le ecuación FK}

\begin{equation}
	\label{formularecursiva}
	\displaystyle \begin{cases} w_{_{i,j+1}}=w_{_{i,j}}+\sigma(w_{_{i-1,j}}-2w_{_{i,j}}+w_{_{i+1,j}})+k\rho(A-w_{_{i,j}})w_{_{i,j}}; \,\,1\leq i<m
		
	\vspace{0.5cm} \\
	
	w_{_{0,j+1}}=w_{_{0,j}}+\sigma(2w_{_{1,j}}-2w_{_{0,j}})+k\rho(A-w_{_{0,j}})w_{_{0,j}}
	
	\vspace{0.5cm} \\
	
	
	\vspace{0.5cm}
	w_{_{n,j+1}}=w_{_{n,j}}+\sigma(2w_{_{n-1,j}}-2w_{_{n,j}})+k\rho(A-w_{_{n,j}})w_{_{n,j}}\\
	
	w_{_{i,0}}=g(x_i)\quad 0\leq i\leq m
	\end{cases}
\end{equation}

El siguiente capítulo haremos un programa en Matlab del sistema \eqref{formularecursiva}\\

También podemos expresar \eqref{formularecursiva} dando una forma de iteración matricial. Antes debemos expresar \eqref{iteracionsinneumann} como $$w_{_{i,j+1}}=\sigma w_{_{i-1,j}}+(1-2\sigma)w_{_{i,j}}+\sigma w_{_{i+1,j}}+k\rho(A-w_{_{i,j}})w_{_{i,j}}$$

Luego también considerando \ref{neumannizquierdo} y \eqref{neumannderecha} se tiene:

$$
\begin{bmatrix} 
	w_{_{1,j+1}} \\
	w_{_{2,j+1}} \\
	\vdots\\
	w_{_{m-2,j+1}}\\
	w_{_{m-1,j+1}}
\end{bmatrix}=\begin{bmatrix}
	1-2\sigma  & \sigma    &           &        &      &  &  & \\
	\sigma     & 1-2\sigma & \sigma    &        &      &  &  &  \\
	& \sigma    & 1-2\sigma &\sigma  &      &  &  &  \\
	&           &           &        &\ddots&  &  &  \\
	&           &           &        &\sigma&1-2\sigma&\sigma\\     
	&           &           &        &      &\sigma   &1-2\sigma\\ 			  		     	
\end{bmatrix}\begin{bmatrix}
	w_{_{1,j}} \\
	w_{_{2,j}} \\
	\vdots\\
	w_{_{m-2,j}}\\
	w_{_{m-1,j}}
\end{bmatrix}+\begin{bmatrix}
	\sigma h_{_{1}}(t_{_{j}})+k_{_{1}}(t_{_{j}})\\
	k_{_{2}}(t_{_{j}})\\
	\vdots\\
	k_{_{m-2}}(t_{_{j}})\\
	\sigma h_{_{2}}(t_{_{j}}) + k_{_{m-1}}(t_{_{j}})
\end{bmatrix}		      			  		
$$
donde $k_{_{i}}(t_{_{j}})=k\rho(A-w_{_{i,j}})w_{_{i,j}}$ y $h_{_{1}}(t_{_{j}})=w_{_{0,j}}$ y $h_{_{2}}(t_{_{j}})=w_{_{m,j}}$
 
 Utilizando la condición inicial se tiene que:
 
 $$
 w^{(0)}=\begin{bmatrix}
 	g(x_{_{1}})\\
 	g(x_{_{2}})\\
 	\vdots\\          
 	g(x_{_{m-2}})\\ 
 	g(x_{_{m-1}})
 \end{bmatrix}
 $$
 %%%%%%%%%%%%%%%%%%%%%%%%%%%%%%%%%%%%%%%%%%%%%%%%%%%%%%%%%%%%%%%%%%%%%%%%%%%%%%%%%
 \section{Propiedades de la solución.}
 
 A continuación se enuncian algunas propiedades de la solución de la ecuación de Fisher, en el caso $\rho=d=A=1$.\\
 
\begin{teorem}\label{primerteoremapropiedades}
	\upshape{
		Suponer $u(x,t)$, satisfaciendo $u,u_{_{x}},u_{_{xx}},u_{_{t}}\in C([0,1]\times[0,\infty])$, es la solución de un problema de la forma:
		
		\begin{equation}
			\label{sistteorema}
			\displaystyle \begin{cases} 
				u_{_{t}}=u_{_{xx}}+u(1-u) 
				
				\vspace{0.5cm} \\
				
				u_{_{x}}(0,t)=u_{_{x}}(1,t)=0
				
				\vspace{0.5cm} \\
				
				u(x,0)=f(x)
			\end{cases}
		\end{equation}
	
	Entonces, si la condición inicial $f(x)$ satisface $0<\epsilon\leq f(x)\leq 1+\epsilon$, se cumple que la solución $u(x,t)$ en cualquier instante de tiempo cumple: 
	
	$$0<\epsilon\leq u(x,t)\leq 1+\epsilon,$$
	
	para todo $x\in[0,1]$, $t\geq 0$.
	}
\end{teorem}

Sea $u(x,t)$ la solución de \eqref{sistteorema} con $f(x)$ satisfaciendo $0<\varepsilon\leq f(x)\leq 1+\varepsilon$. Se define para $t\geq 0$:
 
$$E(t)=\int_{0}^{1}(u(x,t)-1)^{2}dx.$$

Usando $u_{_{t}}=u_{_{xx}}+u(1-u)$ y las condiciones de frontera $u_{_{x}}(0,t)=u_{_{x}}(1,t)=0$, se obtiene:

$$E'(t)=2\int_{0}^{1}(u-1)u_{_{t}}dx=2\int_{0}^{1}(u-1)u_{_{xx}}-u(1-u)^{2}dx=-2\int_{0}^{1}(u_{_{x}})^{2}-2\int_{0}^{1}u(1-u)^{2}$$

Se sigue del teorema \ref{primerteoremapropiedades} que $u(x,t)\geq\varepsilon>0,\,\forall\, x\in[0,1],\,\,t\geq 0$ y como consecuencia se tiene 

$$E'(t)\leq -2\varepsilon\int_{0}^{1}(1-u(x,t))^{2}dx=-\varepsilon E(t)$$

\vspace{0.4cm}
Por tanto, la \textit{desigualdad de Gronwall} implica que:
 
$$E(t)\leq e^{-2\,\varepsilon\,t}E(0)$$

de donde se obtiene el resultado.

\begin{teorem}\label{sistteoremaproiedad222}
	\upshape{
	Sea $u(x,t)$ solución del problema \eqref{sistteorema} con $f(x)$ satisfaciendo $0<\varepsilon\leq f(x)\leq 1+\varepsilon$, para todo $x\in[0,1]$. Entonces la solución asintótica de $u(x,t)$ se aproxima a $u=1$ en el sentido de que: \\
	
	$$\int_{0}^{1}(u(x,t)-1)^{2}dx\leq e^{-2\,\varepsilon\,t}\int_{0}^{1}(1-f(x))^{2}dx$$
	
	para $t\geq 0$
}
\end{teorem}

Para el poblema de Neumann \eqref{sistteorema}, no es difícil ver que la solución puede tender a infinito en un tiempo finito para algunos valores concretos de $g$ con $g(u)=u(1-u)$. Notar que si la condición inicial $f$ es constante, por ejemplo, 

\begin{equation}
	f(x)=f_0
	\label{f0}
\end{equation}

para todo $x\in[0,1]$, entonces:

$$u(x,t)=v(t), x\in[0,1], t>0,$$

donde $v$ es la solución de:
\[
\left\{ \begin{array}{lcl}
	v'(t)=g(v)  \\
	&     &         \\
	v(0)=f_{_{0}}
\end{array}
\right.
\]

Por lo tanto, la solución de \eqref{sistteorema} tiende a infinito en un tiempo finito con unas condiciones iniciales que satisfacen \eqref{f0} cuando $f(u)=u^{3}$ y $f_0>0$
entonces la solución será:

$$v(t)=\dfrac{f_{_{0}}}{\sqrt{1-2\,\,f^{2}_{_{0}}}}$$

la cual cumple que:

$$v(t)\longrightarrow\infty\mbox{ cuando } t\rightarrow\dfrac{1}{2f^{2}_{_{0}}}$$

En conclusión, la solución de \eqref{sistteorema} tiende a infinito en un tiempo finito con unas condiciones iniciales que satisfacen \eqref{f0} cuando $g(u)=u^{3}$ y $f_0>0$.
	