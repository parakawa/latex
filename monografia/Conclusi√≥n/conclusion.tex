%%%%%%%%%%%%%%%%%%%%%%%%%%%%%%%%%%%%%%%%%%%%%%%%%%%%%%%%%%%%%%%
\section{Cuando el parámetro $\rho$ de proliferación es constante }
	Analizar las tablas puede resultar tedioso por podemos analizar las gráficas de la Figura \ref{figuraD}.\\

	Se observa  que independientemente del los valores asignados al coeficiente de difusión ambos modelos muestran que en el periodo de un año llegan  a ocupar el máximo del tejido. (Recordar que $u$ se tomo en su forma normalizada).\\
	
	Por el contrario, se ve que cuando el coeficiente de difusión es significativa el número de células se expande en un periodo muy breve  de tiempo cubriendo todo el eje espacial; sin embargo, cuando $d$ es pequeño no la cubre incluso cuando $u=1$ (en máximo al cabo de un año).
	
\section{Cuando el coeficiente $d$ de difusión es constante}
	
	Para este caso analizaremos la Figura \ref{figuraparaRHO} \\
	
	Lo primero que notamos es que en ambos modelos el tumor cubre casi instantáneamente  todo el eje espacial.\\
	
	Si nos fijamos en el parámetro de difusión notaremos que para $rho=0.1$ a pesar de que se difunde rápidamente  apenas cubre la capacidad del tejido quedándose a lo largo del año  en $U=0.008$ en promedio. Sin embargo, para $rho=100$, un valor muy alto de proliferación, no solo se difunde rápidamente, sino que  además en apenas poco tiempo el tumor ya ha cubierto el máximo del tejido siendo este $u=1$.