En la actualidad el \textit{Cáncer} es una de las palabras que más asusta cuando se habla de salud, ya que no se trata únicamente de una enfermedad sino que este término hace referencia a muchas enfermedades que tienen un denominador común: la transformación de la células normales en células cancerosas las cuales se dividen sin control y pueden invadir otros tejidos.\\

En la literatura existen numerosos modelos matemáticos que intentan describir  procesos que ocurren durante el desarrollo del cáncer, algunos de estos decriben: el crecimiento de tumores cancerosos, la respuesta del sistema inmune a la presencia de células cancerosas en el cuerpo, movimiento de las células cancerosas y su propagación en el organismo, control de crecimiento de tumores, modelos de vascularización o angiogénesis de los mismos, entre otros.\\

El estudio  matemático que presentaremos busca en concreto modelar tumores cerebrales. Para ello haremos uso de las ecuaciones en derivadas parciales que nos permitirá introducir un primer modelo sencillo en la cual nuestro principal objetivo es hacer un análisis y solución numérica del modelo aplicado a un caso concreto, gliomas, para el cual haremos uso de Matlab.